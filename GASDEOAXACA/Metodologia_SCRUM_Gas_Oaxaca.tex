\documentclass[12pt,letterpaper]{article}
\usepackage[utf8]{inputenc}
\usepackage[spanish]{babel}
\usepackage{amsmath}
\usepackage{amssymb}
\usepackage{graphicx}
\usepackage{float}
\usepackage{geometry}
\usepackage{fancyhdr}
\usepackage{hyperref}
\usepackage{listings}
\usepackage{xcolor}
\usepackage{array}
\usepackage{longtable}
\usepackage{booktabs}

\geometry{
    left=2.5cm,
    right=2.5cm,
    top=2.5cm,
    bottom=2.5cm
}

\hypersetup{
    colorlinks=true,
    linkcolor=blue,
    filecolor=magenta,
    urlcolor=cyan,
    pdftitle={Metodología SCRUM - Sistema de Gestión de Gas de Oaxaca},
    pdfpagemode=FullScreen,
}

\pagestyle{fancy}
\fancyhf{}
\rhead{Proyecto Gas Oaxaca}
\lhead{Metodología SCRUM}
\cfoot{\thepage}

\lstset{
    basicstyle=\ttfamily\footnotesize,
    breaklines=true,
    frame=single,
    backgroundcolor=\color{gray!10}
}

\title{\textbf{Metodología SCRUM Aplicada al Proyecto}\\
\large Sistema de Gestión de Gas de Oaxaca}

\author{Equipo de Desarrollo}
\date{\today}

\begin{document}

\maketitle
\tableofcontents
\newpage

\section{Introducción}

El presente documento describe la aplicación de la metodología ágil SCRUM en el desarrollo del Sistema de Gestión de Gas de Oaxaca. Este proyecto tiene como objetivo digitalizar y optimizar los procesos de gestión de distribución y comercialización de gas en el estado de Oaxaca, mejorando la eficiencia operativa y la experiencia del cliente.

\subsection{Objetivos del Documento}
\begin{itemize}
    \item Documentar la metodología SCRUM aplicada en el proyecto
    \item Definir los requisitos funcionales y no funcionales
    \item Presentar los casos de uso y historias de usuario
    \item Describir el análisis y diseño del sistema
    \item Establecer las bases para el desarrollo iterativo del proyecto
\end{itemize}

\section{Metodología SCRUM}

\subsection{Descripción General}
SCRUM es un marco de trabajo ágil que permite desarrollar productos complejos de manera iterativa e incremental. Se caracteriza por:
\begin{itemize}
    \item Sprints: Ciclos de desarrollo de 2-4 semanas
    \item Roles claros: Product Owner, Scrum Master, Equipo de Desarrollo
    \item Eventos definidos: Sprint Planning, Daily Scrum, Sprint Review, Sprint Retrospective
    \item Artefactos: Product Backlog, Sprint Backlog, Incremento de Producto
\end{itemize}

\subsection{Aplicación en el Proyecto Gas de Oaxaca}

\subsubsection{Roles Definidos}
\begin{itemize}
    \item \textbf{Product Owner}: Responsable de definir los requisitos y priorizar el Product Backlog
    \item \textbf{Scrum Master}: Facilitador del proceso SCRUM, elimina impedimentos
    \item \textbf{Equipo de Desarrollo}: 5-7 profesionales multidisciplinarios (desarrolladores, diseñadores, analistas)
\end{itemize}

\subsubsection{Duración de Sprints}
\begin{itemize}
    \item Duración establecida: 3 semanas por Sprint
    \item Total de Sprints planificados: 8
    \item Duración total del proyecto: 6 meses
\end{itemize}

\section{Requisitos del Sistema}

\subsection{Requisitos Funcionales}

\subsubsection{Gestión de Clientes}
\begin{enumerate}
    \item \textbf{RF-01}: Registro de nuevos clientes con datos personales y de contacto
    \item \textbf{RF-02}: Actualización de información de clientes existentes
    \item \textbf{RF-03}: Consulta de historial de cliente
    \item \textbf{RF-04}: Eliminación lógica de clientes (baja)
    \item \textbf{RF-05}: Asignación de categorías de cliente (residencial, comercial, industrial)
\end{enumerate}

\subsubsection{Gestión de Pedidos}
\begin{enumerate}
    \item \textbf{RF-06}: Creación de nuevos pedidos de gas
    \item \textbf{RF-07}: Asignación automática de repartidor según zona
    \item \textbf{RF-08}: Seguimiento del estado del pedido en tiempo real
    \item \textbf{RF-09}: Cancelación de pedidos pendientes
    \item \textbf{RF-10}: Generación de comprobantes de pedido
\end{enumerate}

\subsubsection{Gestión de Inventario}
\begin{enumerate}
    \item \textbf{RF-11}: Registro de entrada y salida de cilindros de gas
    \item \textbf{RF-12}: Control de niveles de inventario por tamaño de cilindro
    \item \textbf{RF-13}: Alertas automáticas de bajo inventario
    \item \textbf{RF-14}: Reportes de movimiento de inventario
    \item \textbf{RF-15}: Gestión de proveedores y órdenes de compra
\end{enumerate}

\subsubsection{Gestión de Pagos}
\begin{enumerate}
    \item \textbf{RF-16}: Registro de pagos de clientes
    \item \textbf{RF-17}: Múltiples métodos de pago (efectivo, tarjeta, transferencia)
    \item \textbf{RF-18}: Generación de facturas y recibos
    \item \textbf{RF-19}: Control de cuentas por cobrar
    \item \textbf{RF-20}: Reportes de cobranza
\end{enumerate}

\subsubsection{Gestión de Rutas y Repartidores}
\begin{enumerate}
    \item \textbf{RF-21}: Optimización automática de rutas de entrega
    \item \textbf{RF-22}: Asignación de pedidos a repartidores
    \item \textbf{RF-23}: Seguimiento GPS de repartidores en tiempo real
    \item \textbf{RF-24}: Registro de entregas realizadas
    \item \textbf{RF-25}: Gestión de devoluciones y reintegros
\end{enumerate}

\subsubsection{Reportes y Analítica}
\begin{enumerate}
    \item \textbf{RF-26}: Reportes de ventas por período
    \item \textbf{RF-27}: Análisis de rentabilidad por cliente
    \item \textbf{RF-28}: Dashboard con métricas clave (KPIs)
    \item \textbf{RF-29}: Exportación de reportes a PDF y Excel
    \item \textbf{RF-30}: Análisis predictivo de demanda
\end{enumerate}

\subsection{Requisitos No Funcionales}

\subsubsection{Rendimiento}
\begin{itemize}
    \item \textbf{RNF-01}: El sistema debe responder en menos de 3 segundos para operaciones críticas
    \item \textbf{RNF-02}: Soportar hasta 100 usuarios concurrentes sin degradación del servicio
    \item \textbf{RNF-03}: Procesamiento de batch completado en menos de 30 minutos
\end{itemize}

\subsubsection{Seguridad}
\begin{itemize}
    \item \textbf{RNF-04}: Autenticación de usuarios con doble factor
    \item \textbf{RNF-05}: Encriptación de datos sensibles (AES-256)
    \item \textbf{RNF-06}: Auditoría completa de todas las operaciones
    \item \textbf{RNF-07}: Cumplimiento con estándares de protección de datos personales
\end{itemize}

\subsubsection{Disponibilidad}
\begin{itemize}
    \item \textbf{RNF-08}: Disponibilidad del sistema del 99.5\% en horario laboral
    \item \textbf{RNF-09}: Plan de recuperación ante desastres con RTO < 4 horas
    \item \textbf{RNF-10}: Backups automáticos diarios con retención de 30 días
\end{itemize}

\subsubsection{Usabilidad}
\begin{itemize}
    \item \textbf{RNF-11}: Interfaz intuitiva con curva de aprendizaje < 2 horas
    \item \textbf{RNF-12}: Responsive design para dispositivos móviles
    \item \textbf{RNF-13}: Accesibilidad WCAG 2.1 nivel AA
\end{itemize}

\subsubsection{Escalabilidad}
\begin{itemize}
    \item \textbf{RNF-14}: Capacidad para crecer 300\% en volumen de transacciones
    \item \textbf{RNF-15}: Arquitectura microservicios para escalado horizontal
    \item \textbf{RNF-16}: Soporte para expansión a nuevas sucursales
\end{itemize}

\section{Casos de Uso}

\subsection{Actores del Sistema}
\begin{itemize}
    \item \textbf{Cliente}: Usuario final que solicita servicio de gas
    \item \textbf{Operador}: Personal que gestiona pedidos y clientes
    \item \textbf{Repartidor}: Personal que realiza entregas
    \item \textbf{Administrador}: Usuario con permisos completos del sistema
    \item \textbf{Sistema}: Componentes automáticos del software
\end{itemize}

\subsection{Casos de Uso Principales}

\subsubsection{CU-01: Realizar Pedido}
\begin{itemize}
    \item \textbf{Actor}: Cliente
    \item \textbf{Descripción}: El cliente solicita un pedido de gas a través del sistema
    \item \textbf{Precondiciones}: Cliente registrado y autenticado
    \item \textbf{Flujo Normal}:
    \begin{enumerate}
        \item Cliente selecciona tipo y cantidad de gas
        \item Sistema muestra disponibilidad y precio
        \item Cliente confirma dirección de entrega
        \item Cliente selecciona método de pago
        \item Sistema genera pedido y asigna número de seguimiento
        \item Sistema notifica al cliente confirmación del pedido
    \end{enumerate}
    \item \textbf{Postcondiciones}: Pedido creado y en estado "Pendiente"
\end{itemize}

\subsubsection{CU-02: Gestionar Inventario}
\begin{itemize}
    \item \textbf{Actor}: Operador
    \item \textbf{Descripción}: El operador gestiona los niveles de inventario de cilindros
    \item \textbf{Precondiciones}: Operador con permisos de inventario
    \item \textbf{Flujo Normal}:
    \begin{enumerate}
        \item Operador consulta niveles actuales de inventario
        \item Sistema muestra alertas de bajo stock si existen
        \item Operador registra entrada de nuevo inventario
        \item Sistema actualiza niveles y genera reporte
        \item Si es necesario, operador genera orden de compra
    \end{enumerate}
    \item \textbf{Postcondiciones}: Inventario actualizado y sistema sincronizado
\end{itemize}

\subsubsection{CU-03: Optimizar Rutas}
\begin{itemize}
    \item \textbf{Actor}: Sistema
    \item \textbf{Descripción}: El sistema calcula y asigna rutas óptimas de entrega
    \item \textbf{Precondiciones}: Pedidos pendientes de asignación
    \item \textbf{Flujo Normal}:
    \begin{enumerate}
        \item Sistema agrupa pedidos por zona geográfica
        \item Sistema calcula ruta más eficiente usando algoritmos de optimización
        \item Sistema asigna rutas a repartidores disponibles
        \item Sistema genera hoja de ruta para cada repartidor
        \item Sistema notifica a repartidores sus asignaciones
    \end{enumerate}
    \item \textbf{Postcondiciones}: Rutas asignadas y repartidores notificados
\end{itemize}

\subsubsection{CU-04: Registrar Entrega}
\begin{itemize}
    \item \textbf{Actor}: Repartidor
    \item \textbf{Descripción}: El repartidor confirma la entrega de un pedido
    \item \textbf{Precondiciones}: Repartidor con pedidos asignados
    \item \textbf{Flujo Normal}:
    \begin{enumerate}
        \item Repartidor localiza cliente usando GPS del sistema
        \item Repartidor realiza entrega del producto
        \item Repartidor confirma entrega en aplicación móvil
        \item Repartidor captura evidencia (foto o firma)
        \item Sistema actualiza estado del pedido a "Entregado"
        \item Sistema envía notificación al cliente
        \item Sistema actualiza inventario automáticamente
    \end{enumerate}
    \item \textbf{Postcondiciones}: Pedido completado e inventario actualizado
\end{itemize}

\section{Historias de Usuario}

Las historias de usuario se han organizado por épicas following el formato SCRUM:

\subsection{Épica: Gestión de Clientes}

\subsubsection{Sprint 1}
\begin{itemize}
    \item \textbf{HU-001}: "Como cliente nuevo, quiero registrarme en el sistema para poder realizar pedidos de gas de manera rápida y sencilla."
    \item \textbf{Criterios de Aceptación}:
    \begin{itemize}
        \item Formulario de registro con validación de campos
        \item Verificación de correo electrónico
        \item Confirmación de registro exitoso
        \item Creación automática de perfil de cliente
    \end{itemize}

    \item \textbf{HU-002}: "Como cliente registrado, quiero actualizar mi información de contacto y dirección para mantener mis datos actualizados."
    \item \textbf{Criterios de Aceptación}:
    \begin{itemize}
        \item Acceso a perfil de usuario
        \item Formulario editable con validación
        \item Confirmación de cambios
        \item Historial de modificaciones
    \end{itemize}
\end{itemize}

\subsection{Épica: Proceso de Pedidos}

\subsubsection{Sprint 2}
\begin{itemize}
    \item \textbf{HU-003}: "Como cliente, quiero seleccionar el tipo y cantidad de gas que necesito para personalizar mi pedido según mis requerimientos."
    \item \textbf{Criterios de Aceptación}:
    \begin{itemize}
        \item Catálogo visual de productos disponibles
        \item Selección de cantidad deseada
        \item Cálculo automático de precio total
        \item Validación de stock disponible
    \end{itemize}

    \item \textbf{HU-004}: "Como cliente, quiero programar la fecha y hora de entrega para recibir mi gas cuando me convenga."
    \item \textbf{Criterios de Aceptación}:
    \begin{itemize}
        \item Calendario interactivo de disponibilidad
        \item Selección de franjas horarias
        \item Confirmación inmediata de disponibilidad
        \item Recordatorio automático antes de la entrega
    \end{itemize}
\end{itemize}

\subsubsection{Sprint 3}
\begin{itemize}
    \item \textbf{HU-005}: "Como operador, quiero ver todos los pedidos pendientes en un dashboard para gestionarlos eficientemente."
    \item \textbf{Criterios de Aceptación}:
    \begin{itemize}
        \item Vista centralizada de todos los pedidos
        \item Filtros por estado, cliente y fecha
        \item Acciones rápidas de gestión
        \item Indicadores visuales de urgencia
    \end{itemize}

    \item \textbf{HU-006}: "Como cliente, quiero rastrear mi pedido en tiempo real para saber cuándo llegará."
    \item \textbf{Criterios de Aceptación}:
    \begin{itemize}
        \item Mapa con ubicación del repartidor
        \item Tiempo estimado de llegada
        \item Actualizaciones automáticas de estado
        \item Notificaciones push de progreso
    \end{itemize}
\end{itemize}

\subsection{Épica: Pagos y Facturación}

\subsubsection{Sprint 4}
\begin{itemize}
    \item \textbf{HU-007}: "Como cliente, quiero pagar mi pedido usando diferentes métodos para tener flexibilidad en mis transacciones."
    \item \textbf{Criterios de Aceptación}:
    \begin{itemize}
        \item Múltiples opciones de pago (tarjeta, efectivo, transferencia)
        \item Pasarela de pago segura
        \item Confirmación instantánea de transacción
        \item Recibo electrónico generado automáticamente
    \end{itemize}

    \item \textbf{HU-008}: "Como administrador, quiero generar facturas fiscales automáticamente para cumplir con las regulaciones contables."
    \item \textbf{Criterios de Aceptación}:
    \begin{itemize}
        \item Generación de facturas con datos fiscales
        \item Integración con sistema tributario
        \item Envío automático al cliente
        \item Archivo histórico de facturación
    \end{itemize}
\end{itemize}

\subsection{Épica: Logística y Entrega}

\subsubsection{Sprint 5}
\begin{itemize}
    \item \textbf{HU-009}: "Como repartidor, quiero ver mi ruta diaria optimizada en mi móvil para entregas eficientes."
    \item \textbf{Criterios de Aceptación}:
    \begin{itemize}
        \item Aplicación móvil con rutas predefinidas
        \item Navegación GPS integrada
        \item Lista de entregas del día
        \item Estados de entrega actualizables
    \end{itemize}

    \item \textbf{HU-010}: "Como sistema, quiero optimizar automáticamente las rutas de entrega para minimizar tiempo y combustible."
    \item \textbf{Criterios de Aceptación}:
    \begin{itemize}
        \item Algoritmo de optimización de rutas
        \item Consideración de tráfico en tiempo real
        \item Agrupación geográfica de entregas
        \item Balance de carga entre repartidores
    \end{itemize}
\end{itemize}

\section{Análisis del Sistema}

\subsection{Análisis de Stakeholders}

\subsubsection{Stakeholders Primarios}
\begin{itemize}
    \item \textbf{Gerencia de Operaciones}: Necesita visibilidad completa del negocio
    \item \textbf{Equipo de Ventas}: Requiere herramientas de gestión de clientes
    \item \textbf{Personal de Logística}: Necesita optimización de rutas
    \item \textbf{Clientes}: Experiencia simple y confiable
\end{itemize}

\subsubsection{Stakeholders Secundarios}
\begin{itemize}
    \item \textbf{Proveedores}: Integración con sistema de pedidos
    \item \textbf{Entidades Regulatorias}: Cumplimiento normativo
    \item \textbf{Socio Tecnológico}: Mantenimiento y soporte
\end{itemize}

\subsection{Análisis de Riesgos}

\subsubsection{Riesgos Técnicos}
\begin{itemize}
    \item \textbf{Alto}: Integración con sistemas legacy existentes
    \item \textbf{Medio}: Rendimiento con alto volumen de transacciones
    \item \textbf{Bajo}: Compatibilidad con diferentes dispositivos móviles
\end{itemize}

\subsubsection{Riesgos Operacionales}
\begin{itemize}
    \item \textbf{Alto}: Resistencia al cambio por parte del personal
    \item \textbf{Medio}: Disponibilidad de conexión a internet en zonas rurales
    \item \textbf{Bajo}: Capacitación del equipo de repartidores
\end{itemize}

\subsection{Análisis de Requisitos}

\subsubsection{Matriz de Trazabilidad}
\begin{table}[H]
\centering
\small
\begin{tabular}{|l|c|c|c|c|c|}
\hline
\textbf{Requisito} & \textbf{Módulo} & \textbf{Prioridad} & \textbf{Complejidad} & \textbf{Sprint} & \textbf{HU} \\
\hline
RF-01 & Clientes & Alta & Media & 1 & HU-001 \\
RF-06 & Pedidos & Alta & Media & 2 & HU-003 \\
RF-11 & Inventario & Media & Alta & 3 & HU-005 \\
RF-16 & Pagos & Alta & Alta & 4 & HU-007 \\
RF-21 & Logística & Alta & Alta & 5 & HU-009 \\
\hline
\end{tabular}
\caption{Matriz de trazabilidad de requisitos}
\end{table}

\section{Diseño del Sistema}

\subsection{Arquitectura General}

\subsubsection{Arquitectura Microservicios}
El sistema implementa una arquitectura de microservicios con los siguientes componentes:

\begin{itemize}
    \item \textbf{API Gateway}: Punto único de entrada y enrutamiento
    \item \textbf{Servicio de Autenticación}: Gestión de usuarios y seguridad
    \item \textbf{Servicio de Clientes}: CRUD de clientes y perfiles
    \item \textbf{Servicio de Pedidos}: Gestión del ciclo de vida de pedidos
    \item \textbf{Servicio de Inventario}: Control de stock y almacenes
    \item \textbf{Servicio de Pagos}: Procesamiento de transacciones
    \item \textbf{Servicio de Logística}: Optimización de rutas
    \item \textbf{Servicio de Notificaciones}: Email, SMS, push
    \item \textbf{Servicio de Reportes}: Analítica y dashboards
\end{itemize}

\subsubsection{Diagrama de Arquitectura}
\begin{figure}[H]
\centering
% Aquí iría un diagrama de arquitectura
\caption{Arquitectura de microservicios del sistema}
\end{figure}

\subsection{Diseño de Base de Datos}

\subsubsection{Modelo de Datos Principal}
Las entidades principales del sistema incluyen:

\begin{itemize}
    \item \textbf{Cliente}: Información personal y de contacto
    \item \textbf{Pedido}: Cabecera de pedidos con estado y seguimiento
    \item \textbf{DetallePedido}: Productos y cantidades por pedido
    \item \textbf{Producto}: Tipos de gas y presentación
    \item \textbf{Inventario}: Stock disponible por almacén
    \item \textbf{Repartidor}: Información de personal de entrega
    \item \textbf{Ruta}: Asignación de entregas por repartidor
    \item \textbf{Pago}: Transacciones y métodos de pago
\end{itemize}

\subsection{Diseño de Interfaces}

\subsubsection{Principios de Diseño}
\begin{itemize}
    \item \textbf{Mobile First}: Diseño optimizado para dispositivos móviles
    \item \textbf{Responsive}: Adaptación a diferentes tamaños de pantalla
    \item \textbf{Accesible}: Cumplimiento WCAG 2.1
    \item \textbf{Intuitivo}: Minimalismo y claridad en la interfaz
\end{itemize}

\subsubsection{Componentes de UI Principales}
\begin{itemize}
    \item Dashboard principal con KPIs
    \item Formularios de registro y gestión
    \item Mapas interactivos para seguimiento
    \item Tablas con filtros y búsqueda
    \item Componentes de notificación
\end{itemize}

\section{Planificación de Sprints}

\subsection{Sprint 0: Configuración Inicial}
\begin{itemize}
    \item Configuración del entorno de desarrollo
    \item Definición de herramientas y pipeline CI/CD
    \item Configuración de repositorios y documentación
    \item Establecimiento de métricas y KPIs
\end{itemize}

\subsection{Sprint 1-2: Módulo de Clientes}
\begin{itemize}
    \item Implementación de autenticación y autorización
    \item CRUD completo de clientes
    \item Validación de datos y reglas de negocio
    \item Pruebas unitarias y de integración
\end{itemize}

\subsection{Sprint 3-4: Módulo de Pedidos}
\begin{itemize}
    \item Creación y gestión de pedidos
    \item Estados y transiciones del pedido
    \item Notificaciones automáticas
    \item Integración con inventario
\end{itemize}

\subsection{Sprint 5-6: Módulo de Pagos}
\begin{itemize}
    \item Integración con pasarelas de pago
    \item Generación de facturas
    \item Gestión de cuentas por cobrar
    \item Reportes de transacciones
\end{itemize}

\subsection{Sprint 7-8: Módulo de Logística}
\begin{itemize}
    \item Optimización de rutas
    \item Aplicación móvil para repartidores
    \item Seguimiento GPS en tiempo real
    \item Dashboard de gestión logística
\end{itemize}

\section{Métricas y KPIs}

\subsection{Métricas de Proceso}
\begin{itemize}
    \item \textbf{Velocity}: Puntos de historia por Sprint
    \item \textbf{Burndown Chart}: Progreso vs. planificado
    \item \textbf{Cycle Time}: Tiempo desde inicio hasta entrega
    \item \textbf{Lead Time}: Tiempo desde solicitud hasta entrega
\end{itemize}

\subsection{Métricas de Calidad}
\begin{itemize}
    \item \textbf{Coverage}: Porcentaje de código cubierto por pruebas (>80\%)
    \item \textbf{Bug Density}: Defectos por línea de código
    \item \textbf{Code Review}: 100\% de código revisado
    \item \textbf{Technical Debt}: Seguimiento y reducción continua
\end{itemize}

\section{Gestión de Cambios}

\subsection{Proceso de Gestión de Cambios}
\begin{enumerate}
    \item Solicitud de cambio documentada
    \item Análisis de impacto y costo-beneficio
    \item Aprobación del Product Owner
    \item Actualización del Product Backlog
    \item Priorización y asignación a Sprint
\end{enumerate}

\subsection{Control de Versiones}
\begin{itemize}
    \item Git como sistema de control de versiones
    \item Branching strategy: GitFlow
    \item Code reviews obligatorios
    \item Integración continua
\end{itemize}

\section{Conclusiones}

La aplicación de la metodología SCRUM en el proyecto del Sistema de Gestión de Gas de Oaxaca proporciona:

\begin{itemize}
    \item Flexibilidad para adaptarse a cambios en los requisitos
    \item Entrega incremental de valor al negocio
    \item Transparencia y visibilidad del progreso
    \item Mejora continua a través de retrospectivas
    \item Reducción de riesgos mediante iteraciones cortas
\end{itemize}

La planificación detallada de sprints, la definición clara de historias de usuario y la estructura de microservicios establecen las bases para un desarrollo exitoso que cumpla con los objetivos de negocio y técnicos del proyecto.

\section{Anexos}

\subsection{Glosario de Términos}
\begin{itemize}
    \item \textbf{Sprint}: Iteración de desarrollo en SCRUM
    \item \textbf{Backlog}: Lista priorizada de trabajo pendiente
    \item \textbf{User Story}: Requisito desde perspectiva del usuario
    \item \textbf{Velocity}: Capacidad del equipo por Sprint
    \item \textbf{Definition of Done}: Criterios de finalización
\end{itemize}

\subsection{Plantillas}
\begin{itemize}
    \item Plantilla de User Story
    \item Checklist de Definition of Done
    \item Template de Sprint Retrospective
    \item Formato de Daily Scrum
\end{itemize}

\end{document}